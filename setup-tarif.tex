%&тех
\hsize=\pdfpagewidth \advance\hsize by-3cm
\pdfhorigin=2.5cm \hoffset=\pdfpagewidth \advance\hoffset-\hsize \advance\hoffset-2\pdfhorigin
\output={\ifodd\pageno\hoffset=0pt\fi \plainoutput}

\def\today{\number\year-\ifnum\month<10 0\fi\number\month-\ifnum\day<10 0\fi\number\day}
\def\hours{\count255=\time \divide\count255 by60 \ifnum\count255<10 0\fi\number\count255
  \multiply\count255 by-60 \advance\count255 by\time:\ifnum\count255<10 0\fi\number\count255}

\def\rectangle#1#2#3#4{\vbox to0pt{\vss\hrule\kern-.3pt
  \hbox to#3{\vrule height#1 depth#2\hss#4\hss\vrule}%
  \kern-.3pt\hrule\kern-#2\kern-.1pt}}

\footline={\hss\rectangle{8pt}{2pt}{25pt}{\tenrm\folio}\hss}

\def\uncatcodespecials{\def\do##1{\catcode`##1=12}\dospecials}

\catcode`\"=\active
\def"{\leavevmode\hbox\bgroup\let"=\egroup 
      \def\par{\errmessage{\string\par\space inside \string"...\string"}}%
      \uncatcodespecials\tt}

{\obeyspaces\global\let =\ } % let active space = control space

\newcount\ttlineno
\def\begintt{\ttlineno=0
  \everypar={\advance\ttlineno by1 \rlap{\kern13pt\llap{\the\ttlineno:}}}\begtt}
\def\begtt{%
  \smallskip\hrule\nobreak\smallskip\bgroup
  \uncatcodespecials \catcode`\^^M=12 \catcode`\"=12 \obeyspaces
  \def\ttstrut{\vrule height8pt depth3pt width0pt}%
  \offinterlineskip \parskip=-1pt
  \hfuzz=4em
  %%%%%%%%%%%%%
  \advance\hfuzz by1.4pt
  %%%%%%%%%%%%%
  \ttsmall
  \startverb}

{\catcode`\|=0 \catcode`\^^M=12 \catcode`\\=12 %
  |gdef|startverb^^M#1\endtt{|runttloop#1|end^^M}} %
{\catcode`\^^M=12 %
  \gdef\runttloop#1^^M{\ifx\end#1 \expandafter\endttloop %
     \else\noindent %
       \kern17pt %
       \ttstrut#1\par\penalty12 %
       \expandafter\runttloop\fi}} %
\def\endttloop#1{\egroup\everypar={}\nobreak\smallskip\hrule\smallskip}

\def\sec#1\par{\advance\secnum by1 \subsecnum=0
  \vskip18pt plus6pt minus6pt
  \noindent\rectangle{16pt}{9pt}{25pt}{\bbbf\the\secnum}%
  \kern5pt{\bbbf #1}\par\nobreak\medskip}

\def\subsec#1\par{\advance\subsecnum by1
  \removelastskip \bigskip
  \noindent\vbox to0pt{\vss
    \rectangle{16pt}{9pt}{25pt}{\bf\the\secnum.\the\subsecnum}\kern-5pt}%
  \kern5pt{\bbf #1}\par\nobreak\smallskip}

\input epsf

\parindent=30pt
\raggedbottom
\exhyphenpenalty=10000

\font\bbf=omb10 at12pt % subsection
\font\bbbf=omb10 at14.4pt % section
\font\rmsmall=omr8
\font\slsmall=omsl8
\font\ttsmall=omtt8

\newcount\secnum
\newcount\subsecnum

\def\raggedcenter{\leftskip=0pt plus4em \rightskip=\leftskip
  \parfillskip=0pt \spaceskip=.3333em \xspaceskip=.5em
  \pretolerance=9999 \tolerance=9999 \parindent=0pt
  \hyphenpenalty=9999 \exhyphenpenalty=9999 }

\rightline{\rmsmall\today\ \hours}
\nobreak\medskip
\centerline{\bbbf Установка и настройка системы тарификации}
\nobreak
\vskip18pt plus6pt minus6pt

\moveright3.7cm\vbox{\hsize10cm\raggedcenter\it Кроме самого сервера, данные настройки необходимо выполнять и на компьютере разработчика, т.к. перед обновлением программы на сервере изменения нужно протестировать.}
\bigskip

\newcount\n
\def\N{\advance\n by1\indent\hbox to0pt{\hskip-\parindent\bf\the\n.\hfil}}

\sec Обзор системы тарификации

$$\epsfxsize 15cm \epsfbox{setup-tarif.eps}$$
TODO: переделать рисунок и fire.mp - теперь сервер в 40-й сети: из asu-100 открыт 22 порт,
на asu-100 открыт 80 порт
\parskip=5pt

\noindent
Система тарификации имеет модульную структуру. Она включает в себя:
\item{$\diamond$} база данных {\it PostgreSQL}
\item{$\diamond$} web-интерфейс для доступа к базе данных
\item{$\diamond$} компьютеры сбора и буферизации тарификационных данных с телефонных станций

\noindent
Кроме основных компонентов, существует ряд вспомогательных программ для выполнения следующих функций: периодический сбор тарификационных данных с буферных компьютеров и занесение их в базу данных, конвертация данных из двоичного формата М-200 в текстовый формат, выбока данных через web-интерфейс из базы данных по запросу пользователя и подсчет стоимости звонков, формирование отчёта для пользователя в формате {\it Excel}.

\noindent
Система тарификации является распределённой: сбор данных происходит с удалённых станций, соединённых компьютерной сетью.
Компьютер сбора тарификации обязательно нужно подключать отдельно к каждой станции с которой выполняется сбор тарификации. Его необходимо подключать либо через COM-порт, либо через {\sl cross-over} кабель по ethernet (чтоб соединение было всегда точка-точка). Это необходимо для того, чтобы если возникнут неполадки в сети, данные не пропали.

\noindent
Программа занесения собранных с каждой станции данных периодически выполняется на центральном сервере.

\noindent
Доступ к тарификации осуществляется через Web-интерфейс по следующим адресам:
\item{} {\tt https://192.168.8.4/tarif} \quad {\rmsmall получение информации по произвольным параметрам}
\item{} {\tt https://192.168.8.4/otdel} \quad {\rmsmall получение информации по отделам}

\vfill
\eject
\sec Тарификационная база данных

\subsec Настройка операционной системы

\medskip

Здесь описывается настройка компьютера, не относящаеся непосредственно к программам для тарификации; она играет вспомогательную
роль и используется последующими скриптами для того, чтобы отличить сервер от тестируемой машины, так как требуется выполнять
разные действия в зависимости от того на какой машине они выполняются, и при этом пользоваться везде одним и тем же скриптом.
\smallskip

\N
Установить ОС (Debian 9) на сервере---при установке выбирать английский язык и английскую
раскладку,
стандартный пакет программ, и задать настройки сети (имя компьютера---``"ats"'').

\noindent Установить необходимые программы:
\begintt
sudo apt-get install \
  ntp \
  gcc \
  gdb \
  strace \
  openssh-server \
  cifs-utils \
  libpcre2-dev \
  screen \
  libdatetime-perl \
  libdatetime-format-natural-perl \
  libexcel-writer-xlsx-perl
\endtt
\baselineskip=9pt\rmsmall
\noindent
1) -- команда для установки пакетов;
2) -- пакет NTP сервера;
3) -- компилятор для сборки в исполняемую форму программ сбора тарификации из исходных кодов;
4) -- отладчик для компилятора;
5) -- программы для трассировки выполнения программ;
6) -- пакет SSH сервера;
7) -- программы для подключения сетевых дисков;
8) -- заголовочные файлы для компилирования программ, использующих библиотеку {\slsmall pcre\/};
9) -- программа для запуска долговременных процессов, которые выдают данные на терминал;
10,11) -- библиотеки {\slsmall perl\/} для скриптов обработки данных.
\par\normalbaselines\rm
\medskip

\noindent Отредактировать "/etc/ssh/sshd_config" следующим образом:
\begtt
-#UseDNS no
+UseDNS no
\endtt
\medskip

\noindent Отредактировать "/etc/ntp.conf" следующим образом:
\begtt
-pool 0.debian.pool.ntp.org iburst
-pool 1.debian.pool.ntp.org iburst
-pool 2.debian.pool.ntp.org iburst
-pool 3.debian.pool.ntp.org iburst
-
+#pool 0.debian.pool.ntp.org iburst
+#pool 1.debian.pool.ntp.org iburst
+#pool 2.debian.pool.ntp.org iburst
+#pool 3.debian.pool.ntp.org iburst
+server 192.168.40.77
\endtt
\medskip

\noindent Добавить пользователя {\sl user\/} в группы:
\begtt
sudo adduser user staff
sudo adduser user adm
\endtt
\medskip
\noindent Перезагрузиться.
\medskip

\subsec Создание базы данных

\medskip

Здесь говорится как установить и настроить сервер базы данных {\sl PostgreSQL}.
\smallskip

\N
Установить систему управления базами данных (СУБД), а также некоторые дополнительные программы для работы с ней. \par
\noindent {\tenbf ЗАМЕЧАНИЕ:} если будет выдано предупреждение, что некоторые пакеты уже установлены, не обращать внимания -- значит эти пакеты нужны для других целей; здесь для полноты мы пишем весь список нужных программ.
\begintt
sudo apt-get install \
  postgresql \
  postgresql-doc \
  libpq-dev \
  libdbd-pg-perl \
  libdbix-class-perl
\endtt
\baselineskip=9pt\rmsmall
\noindent
1) -- команда для установки пакетов;
2) -- пакет сервера базы данных;
3) -- документация для него;
4) -- заголовочные файлы для компилирования программ, работающих с базой данных;
5) -- библиотека {\slsmall perl\/} для обращения к базе;
6) -- библиотека {\slsmall perl\/} для работы с базой, используемая программами web-интерфейса.
\par\normalbaselines\rm
\medskip

\N
Настроить формат даты в PostgreSQL.
\begintt
sudo perl -i -pe "s/datestyle = 'iso, mdy'/datestyle = 'iso, ymd'/" /etc/postgresql/*/main/postgresql.conf \
&& sudo /etc/init.d/postgresql restart
\endtt
\medskip

\N
Создать пользователя {\tt user} и базу данных {\tt mydb}, необходимых для работы системы тарификации. Имя пользователя базы делать таким же как имя системного пользователя, чтобы можно было просто обращаться к базе с командной строки как "psql mydb".
\begintt
sudo -u postgres sh -c 'createuser user && createdb -O user mydb'
\endtt
\medskip

\N
Все подключения к базе данных будут делаться локально через Unix-domain socket, так как база данных находится на той же машине, что и web-сервер. Чтобы web-сервер мог обращаться к базе данных (помимо системного пользователя), нужно указать параметр аутентификации "trust". Для этого файл "pg_hba.conf" нужно изменить следующим образом:
\begtt
-local   all             all                                     peer
+local   all             all                                     trust
\endtt
\noindent Это можно сделать, выполнив следующую команду:
\begintt
sudo perl -i -pe 's/^(local\s+all\s+all\s+)peer$/$1trust/' /etc/postgresql/*/main/pg_hba.conf \
&& sudo /etc/init.d/postgresql restart
\endtt
\medskip

\N
Тарификационные данные содержатся в таблице {\tt cdr}. Создать таблицу {\tt cdr}.
\begintt
psql mydb <<EOF
  CREATE TABLE cdr (
    date_time timestamp not null,
    src varchar(25) not null,
    dst1 varchar(25) not null,
    src2 varchar(25) not null,
    dst varchar(25) not null,
    sec int not null,
    code varchar(4) not null,
    dur int not null,
    in_trunk varchar(4) not null,
    out_trunk varchar(4) not null,
    station_id char(1) not null,
    reserved varchar(100) not null
  );
EOF
\endtt
\medskip

Структура таблицы {\tt cdr}:
\nobreak
$$\vbox{\offinterlineskip \catcode`_=11
\hrule
\halign{\vrule#\tabskip=10pt&\strut\tt#\hfil&\vrule#&\tt#\hfil&\vrule#&\hfil\tt#\hfil&\vrule#\tabskip=0pt\cr
height2pt&\omit&&\omit&&\omit&\cr
&\omit\strut\hidewidth\tt Поле\hidewidth&&\omit\hidewidth\tt Тип\hidewidth&&\omit\tt Свойства&\cr
height2pt&\omit&&\omit&&\omit&\cr
\noalign{\hrule}
height2pt&\omit&&\omit&&\omit&\cr
&date_time&&timestamp without time zone&&not null&\cr
&src&&character varying(25)&&not null&\cr
&dst1&&character varying(25)&&not null&\cr
&src2&&character varying(25)&&not null&\cr
&dst&&character varying(25)&&not null&\cr
&sec&&integer&&not null&\cr
&code&&character varying(4)&&not null&\cr
&dur&&integer&&not null&\cr
&in_trunk&&character varying(4)&&not null&\cr
&out_trunk&&character varying(4)&&not null&\cr
&station_id&&character(1)&&not null&\cr
&reserved&&character varying(100)&&not null&\cr
height2pt&\omit&&\omit&&\omit&\cr}\hrule}$$

Поля "src" и "dst1" содержат номера М-200 до обработки.\par
Поля "src2" и "dst" содержат номера М-200 после обработки.\par
\hangindent=\parindent\hangafter=1 Для станций, не выполняющих изменение номеров в таблице маршрутизации (напр., {\it Avaya\/}), полям {\tt src2} и {\tt dst1} должны присваиваться пустые значения.
\medskip

\N
Создать индексы для таблицы "cdr".
\begintt
psql mydb <<EOF
  CREATE INDEX cdr_date ON cdr (date(date_time));
  CREATE INDEX cdr_src ON cdr (src);
EOF
\endtt
\medskip
\hangindent=\parindent Индекс "cdr_date" ускоряет выполнение запросов по дате (т.е.~по `"date(date_time)"'). \par
\hangindent=\parindent Индекс "cdr_src" ускоряет выполнение запросов, в которых участвует объединение таблиц "cdr" и "sprav".
\bigskip

\subsec Настройка тестовой машины

\medskip

Для более удобной работы с сервером сделаем следующие настройки:
\smallskip

\N
Создать ключ для доступа с тестовой машины на сервер.
\begintt
[ -e ~/.ssh/id_rsa ] || ssh-keygen -t rsa -N "" -f ~/.ssh/id_rsa -q
\endtt
\medskip

\N
Открыть доступ по ключу с тестовой машины на сервер.
\begintt
cat ~/.ssh/id_rsa.pub |
ssh ats "cat >id_rsa.pub
  diff id_rsa.pub .ssh/authorized_keys2 | grep -q ^\< && cat id_rsa.pub >> .ssh/authorized_keys2
  rm id_rsa.pub"
\endtt
\baselineskip=9pt\rmsmall
\noindent
1) -- отправить открытый ключ на компьютере разработчика на стандартный вывод;
2) -- зайти на сервер записать этот открытый ключ во временный файл;
3) -- проверить есть ли уже этот ключ в файле разрешения доступа, и если нет, то добавить ключ в этот файл;
4) -- удалить временный файл.
\par\normalbaselines\rm
\medskip

На сервере и на компьютере разработчика таблица {\tt cdr} заполняется по-разному: на сервере -- автоматически из тарификационных данных, поступающих со станции; на компьютере разработчика такого источника нет -- поэтому таблица {\tt cdr} заполняется искусственными данными.
\smallskip

\N
Очистить таблицу {\tt cdr}.
\begintt
if [ "$(hostname)" = "ats" ]; then
  echo 'This cannot be run on server!'
else
  psql mydb -c 'DELETE FROM cdr'
fi
\endtt
\medskip

\N
Выполнить вставку искусственных данных в таблицу {\tt cdr}.
\begintt
if [ "$(hostname)" = "ats" ]; then
  echo 'This cannot be run on server!'
else
  YESTERDAY=`date -d "-1 day" +%Y-%m-%d`
  psql mydb <<EOF
    INSERT INTO cdr (date_time,src,dst1,src2,dst,sec,code,dur,in_trunk,out_trunk,station_id,reserved)
    VALUES
    ( CAST( '2017-11-24'||' 18:56:00' AS timestamp ),'5210','','','5202','5','','0','','','8',''),
    ( CAST( '2017-11-28'||' 13:52:00' AS timestamp ),'7203','','','7203','0','1','0','7109','812','6','4074'),
    ( CAST( '2017-11-28'||' 09:39:00' AS timestamp ),'7002','','','89130482785','116','0','0','7136','7220','7','7002'),
    ( CAST( '2017-11-27'||' 23:57:05' AS timestamp ),'7689','','','7937','13','','0','','','5',''),
    ( CAST( '$YESTERDAY'||' 09:23:00' AS timestamp ),'39169993867','','','83912441613',48,'9',0,'','5021','0','' ),
    ( CAST( '2009-09-09'||' 09:23:00' AS timestamp ),'3916993233','','','09',48,'9',0,'','5021','2','' ),
    ( CAST( '2009-09-08'||' 09:23:00' AS timestamp ),'3916994567','','','83916938690',48,'9',0,'','5021','0','' ),
    ( CAST( '2009-09-08'||' 09:23:00' AS timestamp ),'3916925734','','','1111',48,'9',0,'','5021','2','' ),
    ( CAST( '2009-09-07'||' 09:23:00' AS timestamp ),'3916994567','','','83916927548',48,'',0,'','5021','0','' ),
    ( CAST( '2009-09-06'||' 09:23:00' AS timestamp ),'5428','','','89134869328',48,'9',0,'','5021','2','' ),
    ( CAST( '2009-09-04'||' 09:23:00' AS timestamp ),'4949','','','89233774829',48,'9',0,'','5021','0','' ),
    ( CAST( '2009-09-04'||' 08:01:00' AS timestamp ),'5678','','','1111',48,'9',0,'','5021','0','' ),
    ( CAST( '2009-09-08'||' 09:23:00' AS timestamp ),'3916994567','','','83916992683',48,'9',0,'','5021','0','' ),
    ( CAST( '2009-09-08'||' 09:23:00' AS timestamp ),'3916920683','','','83916995089',48,'9',0,'','5021','0','' );
EOF
fi
\endtt
\medskip

\vfill
\eject
\sec Обновление таблицы справочника в базе

\bigskip

В базе {\it PostgreSQL\/} хранятся тарификационные данные. В них содержится телефонный номер и данные об этом номере. Для того, чтобы связать тарификацию с данными об абоненте (напр., подразделение) нам необходима в этой же базе ещё одна таблица. В этой таблице будут содержаться справочные данные (данные, не имеющие телефонного номера, не будут попадать в объединение таблиц). Поле ``номер телефона'' выступает в роли ключа, который связывает эти две таблицы.
\medskip

В таблицу {\it PostgreSQL\/} эти данные загружаются из другого источника. В самой базе {\it PostgreSQL\/} справочные данные не меняются (т.к.\ используются только для формирования выборок). Данные меняются только в источнике, поэтому не будет возникать несответствий. В {\it PostgreSQL\/} данные периодически обновляются из источника (делается полная очистка таблицы, и заполнение её новыми данными).
\medskip

\subsec Создание таблицы

\medskip

Данные для таблицы {\tt sprav} берутся из таблицы {\tt Кроссы} в базе Access. Нижеследующие скрипты извлекают данные из этой таблицы, и затем вставляют полученные данные в базу SQL.
\medskip

\N
Установить программу "mdb-export" для извлечения данных из базы Access.
\begtt
sudo apt-get install mdbtools
\endtt
\medskip

\N
Создать таблицу "sprav".
\begintt
psql mydb <<EOF
  CREATE TABLE sprav (
    otdel varchar(255),
    nomer_ta varchar(255)
  );
EOF
\endtt
\medskip

Структура таблицы {\tt sprav}:

$$\vbox{\tabskip=0pt \offinterlineskip \catcode`_=11
\hrule
\halign{\vrule#\tabskip=10pt&\strut\tt#\hfil&\vrule#&\tt#\hfil&\vrule#&\hfil\tt#\hfil&\vrule#\tabskip=0pt\cr
height2pt&\omit&&\omit&&\omit&\cr
&\omit\strut\hidewidth\tt Поле\hidewidth&&\omit\hidewidth\tt Тип\hidewidth&&\omit\tt Свойства&\cr
height2pt&\omit&&\omit&&\omit&\cr
\noalign{\hrule}
height2pt&\omit&&\omit&&\omit&\cr
&otdel&&character varying(255)&&&\cr
&nomer_ta&&character varying(255)&&&\cr
height2pt&\omit&&\omit&&\omit&\cr}\hrule}$$
\bigskip

\subsec Заполнение таблицы

\medskip

Здесь описывается как заполнить таблицу "sprav" данными из файла Access.
\smallskip

\N
Очистить таблицу "sprav".
\begtt
psql mydb -c 'delete from sprav'
\endtt
\medskip

\N
Скопировать файл "АТС архив.mdb" в домашний каталог сервера базы данных через WinSCP с ASU-100.
Файл находится по адресу \hfil\break
\centerline{\tt Диск S:/Ресурсы подразделений/Отдел телекоммуникационного оборудования/БД/АТС/Архив/}

\N
Экспортировать таблицу {\tt Кроссы} в текстовый файл.
\begtt
mdb-export -Q -H -d'|' ~/"АТС Архив.mdb" Кроссы        >out.txt
mdb-export -Q -H -d'|' ~/"АТС Архив.mdb" Organisations >org.txt
\endtt
\medskip

\N
Подготовить файл {\tt sprav.sql} с SQL-запросами.
\begintt
num=$(mdb-export -Q -d'|' ~/"АТС Архив.mdb" Кроссы |sed -n '1s/|Код подразделения|.*/|/p'|tr -dc '|'|wc -m)
perl -a -F'\|' -nle "print qq(INSERT INTO sprav (otdel,nomer_ta) VALUES ('@F[$num]','@F[3]');)" < out.txt > sprav.sql
perl -a -F'\|' -nle "print qq(INSERT INTO sprav (otdel,nomer_ta) VALUES ('1000','@F[3]');) if @F[3]ne''" \
  < out.txt >> sprav.sql
rm out.txt
grep 'АО ПО ЭХЗ' org.txt|perl -a -F'\|' -nle "if(@F[2]ne''){print qq(s/@F[0]/@F[2]/)}else{print qq(s/@F[0]/0/)}" >org.sed
rm org.txt
sed -i -f org.sed sprav.sql
rm org.sed
sed -i 's/org[0-9]\+//' sprav.sql
\endtt
\medskip

\N
Выполнить вставку новых данных в таблицу {\tt sprav}.
\begintt
psql mydb < sprav.sql
rm sprav.sql
\endtt
\medskip

\N
Создать список подразделений для web-интерфейса.
\begintt
mkdir -p ~/0000-git/
psql mydb -t -c 'select otdel from sprav' | grep -v '^\s*$' | perl -pe 's/^\s+//' | \
perl -e 'print for sort { $a =~ /^\d/ && $b =~ // ? $a <=> $b : $a cmp $b } <>' | \
perl -pe 's/^/<option>/;s!$!</option>!' | uniq > ~/0000-git/otdels.html
\endtt
\bigskip

\noindent
Теперь (учитывая что установлен web-сервер -- см.\ следующий раздел) заходим на \hfil\break
"https://192.168.40.34/otdel" и сохраняем а потом копируем по scp на сервер а потом
с ASU-100 раскладываем по папкам через WinSCP.

\vfill
\eject
\sec Настройка web-сервера

\bigskip

\subsec Настройка операционной системы

\medskip

Здесь описывается как настроить подключение к серверу тарификации с тестового компьютера.
\smallskip

\N
Генерируем ssh-ключи (только на сервере).
\begintt
[ -e ~/.ssh/id_rsa ] || ssh-keygen -t rsa -N "" -f ~/.ssh/id_rsa -q
\endtt
\medskip

\subsec Настройка Apache

\medskip

Здесь описывается как установить и настроить web-сервер.
\smallskip

\N
Устанавливаем web-сервер (Apache), а также некоторые дополнительные пакеты, дающие возможность выполнять на нём программы обработки тарификации.
%%%%%%%%%%%%%%%%%%%%%%%%%%%%%%%%%%%%%%%%%%%%%%%%%%%%%%%%%%%%%%%%%%%%%%%%%%%%%%%%%%%%
% TODO: add package which will install all dependencies which apache2-doc installs %
%%%%%%%%%%%%%%%%%%%%%%%%%%%%%%%%%%%%%%%%%%%%%%%%%%%%%%%%%%%%%%%%%%%%%%%%%%%%%%%%%%%%
\begintt
sudo apt-get install \
  apache2-doc \
  libapache2-mod-perl2 \
  libapache-db-perl \
  libapache-session-perl \
  libtemplate-perl \
  libtemplate-perl-doc \
  libdata-formvalidator-perl \
  libhtml-fillinform-perl \
  libhtml-tokeparser-simple-perl \
  libjson-perl \
  libmime-lite-perl
\endtt
\medskip

\N
Настроить алиасы для простого просмотра логов и для просмотра логов в режиме реального времени.
\medskip
{\tt l} -- просто просмотр логов \par
{\tt t} -- просмотр логов в реальном времени \par
\medskip
\begintt
cat << 'EOF' >> ~/.bashrc
alias l="less +G /var/log/apache2/error.log"
alias t="tail -f -n 0 /var/log/apache2/error.log"
EOF
. ~/.bashrc
\endtt
\medskip

\N
Настроить алиасы для перезапуска и отладки сервера (на компьютере разработчика).
\begintt
cat << 'EOF' >> ~/.bashrc
alias arst="sudo /etc/init.d/apache2 restart"
alias adbg="sudo /etc/init.d/apache2 stop && sudo /usr/sbin/apache2ctl -X -DPERLDB"
EOF
. ~/.bashrc
\endtt
\medskip

\N
Делаем настройки {\tt sudo}, необходимые для работы алиасов (на компьютере разработчика).
\begintt
sudo sh -c 'cat >> /etc/sudoers' << 'EOF'
user ALL = NOPASSWD: /etc/init.d/apache2 stop, /etc/init.d/apache2 restart, /usr/sbin/apache2ctl -X -DPERLDB
EOF
\endtt
\medskip

\N
Создать алиас "arst" для загрузки обновлений программ с тестовой машины на сервер и перезапуска apache (только на сервере). \par
\begintt
cat << 'EOF' >> ~/.bashrc
alias arst='cd && rm -fr 0000-git/code/ && mkdir -p 0000-git/ &&
  scp -o StrictHostKeyChecking=no -o UserKnownHostsFile=/dev/null -r `echo $SSH_CLIENT|cut -d" " -f1`:0000-git/code/ 0000-git/ > /dev/null &&
  sudo /etc/init.d/apache2 restart'
EOF
. ~/.bashrc
\endtt
\medskip

\N
Делаем настройки {\tt sudo}, необходимые для работы алиаса "arst" (только на сервере).
\begintt
sudo sh -c 'cat >> /etc/sudoers' << 'EOF'
user ALL = NOPASSWD: /etc/init.d/apache2 restart
EOF
\endtt
\medskip

\N
Для работы алиаса {\tt arst} без пароля, открыть доступ по ключу с сервера на тестовую машину (на компьютере разработчика).
\begintt
ssh ats cat .ssh/id_rsa.pub >>~/.ssh/authorized_keys2
\endtt
\medskip

\N
Задать путь к статическим ресурсам web-сервера.
\begintt
sudo rm -r /var/www/html/
sudo ln -s /home/user/0000-git/code/www/DOCUMENTS/ /var/www/html
\endtt
\medskip

\N
Создать файл для хранения имён пользователей и их паролей доступа к ресурсам web-интерфейса.
\begintt
sudo touch /var/www/htpasswd
sudo chmod 640 /var/www/htpasswd
sudo chgrp www-data /var/www/htpasswd
\endtt
\medskip

\N
Создать необходимых пользователей.
\begintt
sudo htpasswd /var/www/htpasswd имя_нового_пользователя
\endtt
\medskip

\N
Отключить сжатие выдаваемых страниц для отладки через программу "wireshark" (только на компьютере разработчика).
\begintt
[ "$(hostname)" = "ats" ] || sudo a2dismod deflate
\endtt
\medskip

\N
Настроить web-сервер для работы программы выдачи тарификации через https
(порт 443) и для просмотра директории с файлами через http (порт 80). \par
\noindent {\bf ВНИМАНИЕ:} Заменить {\tt X.X.X.X} на IP-адрес сервера.
\begintt
sudo a2ensite default-ssl
sudo a2enmod ssl
sudo sh -c 'echo ServerTokens Minimal >> /etc/apache2/apache2.conf'
sudo sh -c 'echo KeepAlive Off >> /etc/apache2/apache2.conf'
sudo sh -c 'echo MaxRequestsPerChild 10000 >> /etc/apache2/apache2.conf'
sudo sh -c 'echo AddDefaultCharset UTF-8 >> /etc/apache2/apache2.conf'
sudo perl -i -0777 -pe 's/^/ServerName X.X.X.X\n/' \
  /etc/apache2/sites-enabled/$(readlink /etc/apache2/sites-enabled/default-ssl*)
sudo perl -i -ne 'print,next if s/^\s*SSLSessionCache\s+.*/SSLSessionCache nonenotnull/;print' \
  /etc/apache2/mods-available/ssl.conf
sudo perl -i -ne 'print,next if 
  s/^\s*SSLCipherSuite\s+.*/SSLCipherSuite ALL:!ADH:!EXPORT56:RC4+RSA:+HIGH:+MEDIUM:+LOW:+SSLv2:+EXP/;print' \
  /etc/apache2/mods-available/ssl.conf
sudo perl -i -ne \
  'print,next if s@^\s*SSLCertificateFile\s+.*@SSLCertificateFile /usr/local/SSL/demoCA/newcert.pem@;print' \
  /etc/apache2/sites-enabled/$(readlink /etc/apache2/sites-enabled/default-ssl*)
sudo perl -i -ne \
  'print,next if s@^\s*SSLCertificateKeyFile\s+.*@SSLCertificateKeyFile /usr/local/SSL/demoCA/newkey.pem@;print' \
  /etc/apache2/sites-enabled/$(readlink /etc/apache2/sites-enabled/default-ssl*)
sudo perl -i -pe 'print "Include /home/user/0000-git/code/www/httpd.conf\n" if m@</VirtualHost>@' \
  /etc/apache2/sites-enabled/$(readlink /etc/apache2/sites-enabled/default-ssl*)
sudo perl -i -pe 's!(?<=DocumentRoot /var/www/)html!dummy!' \
  /etc/apache2/sites-enabled/$(readlink /etc/apache2/sites-enabled/default-ssl*)
\endtt
\medskip

\N
Сгенерировать SSL-сертификаты. Если ранее они уже были сгенерированы и необходимо их заменить, то нужно сначала удалить директорию "/usr/local/SSL/". \par
\noindent {\bf ВНИМАНИЕ:} Заменить {\tt X.X.X.X} на IP-адрес сервера. \par
\begintt
if [ ! -e '/usr/local/SSL/' ]; then
  mkdir /usr/local/SSL/
  cd /usr/local/SSL/
  mkdir demoCA/
  mkdir demoCA/private/
  mkdir demoCA/newcerts/
  touch demoCA/index.txt

  # Certificate authority:
  openssl req -config /etc/ssl/openssl.cnf -new \
    -keyout demoCA/private/cakey.pem -out demoCA/careq.pem -days 730 -nodes \
    -subj "/countryName=RU/stateOrProvinceName=Krasnoyarskiy kray/organizationName=ecp/commonName=CA team"
  openssl ca -config /etc/ssl/openssl.cnf -create_serial -out demoCA/cacert.pem -days 730 \
    -batch -keyfile demoCA/private/cakey.pem -selfsign -extensions v3_ca -infiles demoCA/careq.pem

  # Certificate:
  openssl req -config /etc/ssl/openssl.cnf -new -nodes -keyout demoCA/newkey.pem -out demoCA/newreq.pem \
    -days 730 -subj "/commonName=X.X.X.X"
  openssl ca -config /etc/ssl/openssl.cnf -batch -days 730 \
    -policy policy_anything -out demoCA/newcert.pem -infiles demoCA/newreq.pem

  cd -
fi
\endtt
\medskip

\N
Создать директорию для "tempdir" в "0000-git/code/www/Local_Handler/Otdel.pm":
\begtt
sudo mkdir /tex_tmp/
sudo chmod 777 /tex_tmp/
\endtt
\medskip

\N
Скопировать папку {\tt code/} в {\tt \char`~/0000-git/} на компьютере разработчика, после чего запустить apache с новой конфигурацией.
\begintt
arst
\endtt
\noindent {\tenbf ЗАМЕЧАНИЕ:} на сервере эта команда может выдать подобную ошибку:
\begtt
@@@@@@@@@@@@@@@@@@@@@@@@@@@@@@@@@@@@@@@@@@@@@@@@@@@@@@@@@@@
@    WARNING: REMOTE HOST IDENTIFICATION HAS CHANGED!     @
@@@@@@@@@@@@@@@@@@@@@@@@@@@@@@@@@@@@@@@@@@@@@@@@@@@@@@@@@@@
IT IS POSSIBLE THAT SOMEONE IS DOING SOMETHING NASTY!
Someone could be eavesdropping on you right now (man-in-the-middle attack)!
It is also possible that the RSA host key has just been changed.
The fingerprint for the RSA key sent by the remote host is
0f:8c:74:9c:f9:87:1f:37:f1:42:3a:72:91:0a:f2:96.
Please contact your system administrator.
Add correct host key in /home/user/.ssh/known_hosts to get rid of this message.
Offending key in /home/user/.ssh/known_hosts:3
RSA host key for 192.168.40.190 has changed and you have requested strict checking.
Host key verification failed.
\endtt
\noindent В таком случае нужно вначале удалить идентификатор тестовой машины при помощи следующей команды (на сервере): \par\nobreak
\begtt
sed -i Nd ~/.ssh/known_hosts
\endtt
\noindent Здесь "N" --- это номер строки файла "known_hosts" из сообщения об ошибке (в данном примере "N"="3"). \par
\noindent (Это сообщение выдается когда компьютер, с которого мы подключаемся к серверу, был переустановлен с момента последнего обращения с сервера на него.)

\vfill
\eject
\sec Программы для занесения тарификации в БД

\bigskip

\noindent
Программы устанавливаются на центральном сервере. Они нужны для того, чтобы периодически собирать тарификацию с буферных компьютеров и заносить её в базу данных. Пока информация не занесена в базу данных, она не будет доступна для просмотра.
\smallskip
\noindent
Для каждой станции, с которой необходимо собирать тарификацию, создаётся своя программа. Кроме того, что эти программы заносят информацию в базу данных, некоторые из них ещё и конвертируют данные из формата каждой конкретной станции в общий формат (если используется программа сбора тарификации, поставляемая производителем), перед тем как сохранить их в базу. Благодаря этому система тарификации может быть адаптирована для работы с разными типами телефонных станций, при этом для добавления новой станции нужно лишь создать соответствующую программу.
\bigskip

\noindent
В текущей конфигурации задействованы следующие программы:
\medskip

\noindent $\diamond$ {\tt cdr\char`_write\char`_2.pl}
\item{} - источник сбора данных: М-200 на АТС-2
\item{} - место нахождения собираемой информации: {\tt /media/ats2/tarif/}
\item{} - место нахождения лог-файла: {\tt /var/local/ats2.log}
\smallskip

\noindent $\diamond$ {\tt cdr\char`_write\char`_3.pl}
\item{} - источник сбора данных: Avaya в зд.~26
\item{} - место нахождения собираемой информации: {\tt /media/avaya/3/}
\item{} - место нахождения лог-файла: {\tt /var/local/avaya.log}
\smallskip

\noindent $\diamond$ {\tt cdr\char`_write\char`_4.pl}
\item{} - источник сбора данных: М-200 в зд.~26
\item{} - место нахождения собираемой информации: {\tt /media/upats/tarif/}
\item{} - место нахождения лог-файла: {\tt /var/local/upats.log}
\smallskip

\noindent $\diamond$ {\tt gud}
\item{} - источник сбора данных: Goodwin
\item{} - место нахождения собираемой информации: {\tt /media/goodwin/}
\item{} - место нахождения лог-файла: {\tt /var/local/goodwin.log}
\smallskip

\noindent $\diamond$ {\tt coral-zd5}
\item{} - источник сбора данных: Coral зд.5
\item{} - место нахождения собираемой информации: {\tt /media/coral-zd5/}
\item{} - место нахождения лог-файла: {\tt /var/local/coral-zd5.log}
\bigskip

\noindent $\diamond$ {\tt coral-w}
\item{} - источник сбора данных: Coral w
\item{} - место нахождения собираемой информации: {\tt /media/coral-w/}
\item{} - место нахождения лог-файла: {\tt /var/local/coral-w.log}
\bigskip

\penalty-10
\noindent $\diamond$ {\tt minicom-zd26}
\item{} - источник сбора данных: minicom зд.26
\item{} - место нахождения собираемой информации: {\tt /media/minicom-zd26/}
\item{} - место нахождения лог-файла: {\tt /var/local/minicom-zd26.log}
\bigskip

\noindent $\diamond$ {\tt minicom-zd4}
\item{} - источник сбора данных: minicom зд.4
\item{} - место нахождения собираемой информации: {\tt /media/minicom-zd4/}
\item{} - место нахождения лог-файла: {\tt /var/local/minicom-zd4.log}
\bigskip

\noindent
Первые три скрипта запускаются автоматически при старте системы.
Для их работы нужно добавить пользователя {\sl user\/} в
группу {\sl staff\/}: \hfil\break
\centerline{\tt sudo adduser user staff}
\medskip

\noindent
Для запуска этих скриптов нужно добавить следующие строки в `{\tt crontab -e}'
(они запустятся после перезагрузки): \hfil\break
\indent {\tt @reboot perl /home/user/0000-git/code/cdr\char`_write\char`_2.pl} \hfil\break
\indent {\tt @reboot perl /home/user/0000-git/code/cdr\char`_write\char`_3.pl} \hfil\break
\indent {\tt @reboot perl /home/user/0000-git/code/cdr\char`_write\char`_4.pl} \par
\medskip

\noindent
Также, для этих скриптов необходимо создать директории командой \hfil\break
\centerline{\tt sudo mkdir /media/upats/ /media/ats2/ /media/avaya/}
и добавить в {\tt /etc/fstab} следующие команды для подключения удалённых дисков
по сети: \par
\indent {\tt //192.168.40.141/D /media/upats/ cifs  username=user,password=1,uid=user,port=139 0 0} \hfil\break
\indent {\tt //192.168.40.125/D /media/ats2/ \ cifs  username=user,password=1,uid=user,port=139 0 0} \hfil\break
\indent {\tt //192.168.40.199/D /media/avaya/ cifs  guest,uid=user,port=139 0 0}\par
\medskip

\noindent
Настройки для гудвина, миникома и корала см.\ в документации к программам {\tt gud}, {\tt minicom-zd4}, {\tt minicom-zd26}, {\tt coral-zd5} и {\tt coral-w}.
\bigskip

\noindent Помимо программ занесения данных используются соответствующие программы сбора данных:

\noindent $\diamond$ АТС-2 = {\tt SMPSpider+SMPCallBuilder+call\_build.pl} на 192.168.40.125 \par
\noindent $\diamond$ Avaya = {\tt cdr\_get\_3.pl} на 192.168.40.199 \par
\noindent $\diamond$ upats = {\tt SMPSpider+SMPCallBuilder+call\_build.pl} на 192.168.40.141 \par
\noindent $\diamond$ goodwin = {\tt googwin-log.sh} в кроне на 192.168.40.199 \par
\noindent $\diamond$ Coral w = {\tt cdr-coral-w} в кроне на 192.168.40.53 \par
\noindent $\diamond$ Coral зд.5 = {\tt cdr-coral-zd5} в кроне на 192.168.40.54 \par
\noindent $\diamond$ minicom зд.26 = {\tt DX7term+DX7termRestart.pl} на 192.168.40.50 \par % user SVecp012 (input in remmina - not in prompt)
\noindent $\diamond$ minicom зд.4 = {\tt DX7term+DX7termRestart.pl} на 192.168.40.51 % user 1

\vfill
\eject
\bye
